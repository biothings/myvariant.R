\documentclass[12pt]{article}
%\VignetteIndexEntry{Using MyVariant.R}
%\\SweaveOpts{concordance=TRUE}
\RequirePackage{/Library/Frameworks/R.framework/Versions/3.2/Resources/library/BiocStyle/resources/latex/Bioconductor}

\AtBeginDocument{\bibliographystyle{/Library/Frameworks/R.framework/Versions/3.2/Resources/library/BiocStyle/resources/latex/unsrturl}}
\newcommand{\exitem}[3]
{\item \texttt{\textbackslash#1\{#2\}} #3 \csname#1\endcsname{#2}.}

\title{MyVariant.info R Client}
\author{Adam Mark}

\usepackage{Sweave}
\begin{document}
\Sconcordance{concordance:myvariant.tex:myvariant.Rnw:%
1 2 1 1 2 1 0 1 2 7 1 1 0 17 1 1 4 1 2 1 0 2 1 12 0 1 2 3 1 1 2 1 0 1 1 11 0 1 1 %
3 0 1 2 7 1 1 4 1 2 1 0 1 1 5 0 1 1 6 0 1 2 7 1 1 4 17 0 1 2 11 1 1 2 27 0 1 2 4 %
1 1 2 8 0 1 2 7 1 1 2 1 0 1 1 10 0 1 3 7 1}


\maketitle

\tableofcontents

\section{Overview}

MyVariant.Info is a simple-to-use REST web service to query/retrieve genetic variant annotation from an aggregation of variant annotation resources. \Rpackage{myvariant} is an easy-to-use R wrapper to access MyVariant.Info services and explore variant annotions.
\section{Variant Annotation Service}

\subsection{Obtaining HGVS IDs from a VCF file. }

\begin{itemize}
\item Use \Rfunction{readVcf} from the VariantAnnotation package to read a Vcf file in. The Vcf object can then be passed to \Rfunction{formatHgvs} to retrieve HGVS IDs. HGVS IDs are based on the GRCh38/hg19 reference genome. Support for hg38 is coming soon.
\end{itemize} 


\begin{Schunk}
\begin{Sinput}
> file.path <- system.file("extdata", "dbsnp_mini.vcf", package="myvariant")
> vcf <- readVcf(file.path, genome="hg19")
> rowRanges(vcf)
\end{Sinput}
\begin{Soutput}
GRanges object with 240 ranges and 5 metadata columns:
              seqnames         ranges strand   | paramRangeID            REF
                 <Rle>      <IRanges>  <Rle>   |     <factor> <DNAStringSet>
  rs376643643        1 [10019, 10020]      *   |         <NA>             TA
  rs373328635        1 [10055, 10055]      *   |         <NA>              T
   rs62651026        1 [10108, 10108]      *   |         <NA>              C
  rs376007522        1 [10109, 10109]      *   |         <NA>              A
  rs368469931        1 [10139, 10139]      *   |         <NA>              A
          ...      ...            ...    ... ...          ...            ...
  rs544020171        1 [17654, 17654]      *   |         <NA>              T
  rs563880190        1 [17694, 17694]      *   |         <NA>              C
  rs574335987        1 [17695, 17695]      *   |         <NA>              G
  rs374995955        1 [17697, 17697]      *   |         <NA>              G
  rs543363182        1 [17709, 17709]      *   |         <NA>              T
                             ALT      QUAL      FILTER
              <DNAStringSetList> <numeric> <character>
  rs376643643                  T      <NA>           .
  rs373328635                 TA      <NA>           .
   rs62651026                  T      <NA>           .
  rs376007522                  T      <NA>           .
  rs368469931                  T      <NA>           .
          ...                ...       ...         ...
  rs544020171                  C      <NA>           .
  rs563880190                  T      <NA>           .
  rs574335987                  A      <NA>           .
  rs374995955                  C      <NA>           .
  rs543363182                  G      <NA>           .
  -------
  seqinfo: 1 sequence from hg19 genome; no seqlengths
\end{Soutput}
\end{Schunk}
\begin{itemize}
\item You can then use \Rfunction{formatHgvs} to extract HGVS IDs from the Vcf object.
\end{itemize}

\begin{Schunk}
\begin{Sinput}
> hgvs <- formatHgvs(vcf, variant_type="snp")
> head(hgvs)
\end{Sinput}
\begin{Soutput}
[1] "chr1:g.10108TA>T" "chr1:g.10109T>TA" "chr1:g.10139C>T"  "chr1:g.10150A>T" 
[5] "chr1:g.10177A>T"  "chr1:g.10180TA>T"
\end{Soutput}
\end{Schunk}


\subsection{\Rfunction{getVariant}}

\begin{itemize}
\item Use \Rfunction{getVariant}, the wrapper for GET query of "/v1/variant/<hgvsid>" service, to return the variant object for the given HGVS id.
\end{itemize} 


\begin{Schunk}
\begin{Sinput}
> variant <- getVariant("chr1:g.35367G>A")
> variant[[1]]$dbnsfp$genename
\end{Sinput}
\begin{Soutput}
[1] "FAM138A"
\end{Soutput}
\begin{Sinput}
> variant[[1]]$cadd$phred
\end{Sinput}
\begin{Soutput}
[1] 1.493
\end{Soutput}
\end{Schunk}

\subsection{\Rfunction{getVariants}}

\begin{itemize}
\item Use \Rfunction{getVariants}, the wrapper for POST query of "/v1/variant" service, to return the list of variant objects for the given character vector of HGVS ids.
\end{itemize} 


\begin{Schunk}
\begin{Sinput}
> getVariants(head(hgvs),
+             fields="cadd.consequence")
\end{Sinput}
\begin{Soutput}
DataFrame with 6 rows and 2 columns
   notfound            query
  <logical>      <character>
1      TRUE chr1:g.10108TA>T
2      TRUE chr1:g.10109T>TA
3      TRUE  chr1:g.10139C>T
4      TRUE  chr1:g.10150A>T
5      TRUE  chr1:g.10177A>T
6      TRUE chr1:g.10180TA>T
\end{Soutput}
\end{Schunk}


\section{Variant Query Service}

\subsection{\Rfunction{queryVariant}}

\begin{itemize}
\item \Rfunction{queryVariant} is a wrapper for GET query of "/v1/query?q=<query>" service, to return  the query result. This function accepts wild card input terms and allows you to query for variants that contain a specific annotation. For example, the following query searches for the CADD phred score and consequence for all variants whose genename (dbNSFP) is MLL2.
\end{itemize}



\begin{Schunk}
\begin{Sinput}
> queryVariant(q="dbnsfp.genename:MLL2", fields=c("cadd.phred", "cadd.consequence"))
\end{Sinput}
\begin{Soutput}
$hits
                   _id   _score cadd.consequence cadd.phred
1  chr12:g.49418460A>T 9.948488      STOP_GAINED         59
2  chr12:g.49418658C>A 9.948488      STOP_GAINED         57
3  chr12:g.49420281G>T 9.948488      STOP_GAINED         59
4  chr12:g.49420340T>A 9.948488      STOP_GAINED         53
5  chr12:g.49420364T>A 9.948488      STOP_GAINED         53
6  chr12:g.49420844C>A 9.948488      STOP_GAINED         53
7  chr12:g.49420888G>C 9.948488      STOP_GAINED         53
8  chr12:g.49421602A>T 9.948488      STOP_GAINED         53
9  chr12:g.49421807C>A 9.948488      STOP_GAINED         57
10 chr12:g.49422661T>A 9.948488      STOP_GAINED         45

$max_score
[1] 9.948488

$took
[1] 529

$total
[1] 37953
\end{Soutput}
\end{Schunk}

\begin{itemize}
\item You can also use \Rfunction{queryVariant} to retrieve all annotations that map to a specific rsID.
\end{itemize}

\begin{Schunk}
\begin{Sinput}
> queryVariant(q="rs58991260", fields="dbsnp.flags")$hits
\end{Sinput}
\begin{Soutput}
                  _id   _score                                      flags
1 chr1:g.218631822G>A 17.48191 ASP, G5, G5A, GNO, KGPhase1, KGPhase3, SLO
\end{Soutput}
\end{Schunk}

\subsection{\Rfunction{queryVariants}}

\begin{itemize}
\item \Rfunction{queryVariants} is a wrapper for POST query of "/v1/query?q=<query>" service, to return  the query result. Query terms include any available field as long as scopes are defined. The following example reads the dbSNP rsIDs from a VCF and queries for all fields. The returned DataFrame can then be easily subsetted to include, for example, those that have not been documented in the Wellderly study.
\end{itemize}


\begin{Schunk}
\begin{Sinput}
> rsids <- paste("rs", info(vcf)$RS, sep="")
> res <- queryVariants(q=rsids, scopes="dbsnp.rsid", fields="all")
\end{Sinput}
\begin{Soutput}
Finished
Pass returnall=TRUE to return lists of duplicate or missing query terms.
\end{Soutput}
\begin{Sinput}
> subset(res, !is.na(wellderly.vartype))$query
\end{Sinput}
\begin{Soutput}
 [1] "rs367896724" "rs145427775" "rs147093981" "rs56289060"  "rs55998931"  "rs199606420"
 [7] "rs58108140"  "rs62635284"  "rs62635286"  "rs200579949" "rs531730856" "rs180734498"
[13] "rs527952245" "rs201696125" "rs370886505" "rs546169444" "rs201055865" "rs369473859"
[19] "rs62635298"  "rs79585140"  "rs75454623"  "rs199856693" "rs201855936" "rs71252251" 
[25] "rs201045431" "rs71252250"  "rs200030104" "rs78601809"  "rs62636497"  "rs62636497" 
[31] "rs201635489" "rs533630043" "rs2691315"   "rs113442401" "rs572465511" "rs372319358"
[37] "rs200358166" "rs11489794"  "rs113141985" "rs62636498"  "rs148220436" "rs141130360"
[43] "rs150723783" "rs201459529" "rs199676946" "rs62636367"  "rs62636368"  "rs200205172"
[49] "rs199745162" "rs200658479" "rs201833382" "rs199740902" "rs200978805" "rs201535981"
[55] "rs192890528" "rs200046632" "rs374995955"
\end{Soutput}
\end{Schunk}




\section{References}
MyVariant.info
help@myvariant.info

\end{document}
